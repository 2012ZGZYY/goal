\documentclass{article}
\usepackage{amsmath}
\usepackage{amssymb}

\newcommand{\R}{\mathcal{R}}
\newcommand{\J}{\mathcal{J}}

\title{Adjoint error estimation}
\author{Brian Granzow}
\begin{document}
\maketitle

\section{Introduction}
Error estimation is a key component of numerical
analysis to ensure the accuracy and reliability
of a solution. Often times, numerical studies
are carried out to accurately assess the value
of a functional quantity $\J(u)$. Adjoint-based
error estimation techniques provide a mechanism
to compute error estimates for functional
quantities of interest.

\section{Model problem}
Let $V$ be a Hilbert space equipped with inner
product $(\cdot, \cdot)_V$ and norm $\| \cdot \|_V$.
Let $\R(\cdot; \cdot): V \times V \to \mathbb{R}$ denote
a semilinear form, nonlinear in its first argument
and linear in its second that is associated with
the weak residual form of a partial differential
equation (PDE). Let $u$ denote the exact solution
to the model problem: find $u \in V$ such that
%
\begin{equation}
\R(u;v) = 0 \quad \forall \, u \in V.
\label{eq:model}
\end{equation}

Let $V_H$ and $V_h$ be finite dimensional subspaces
such that $V_H \subset V_h \ \subset V$, where $H$ and
$h$ are positive parameters $0 < h < H$ that indicate
the discretization resolution. A Galerkin
approximation to the model problem \eqref{eq:model}
can be written as: find $u_h \in V_h$ such that
%
\begin{equation}
\R(u_h; v_h) = 0 \quad \forall \, u_h \in V_h,
\label{eq:galerkin}
\end{equation}
where it is assumed that the problem
\eqref{eq:galerkin} is well-posed.

Let $\J(\cdot) : V \to \mathbb{R}$ denote a functional
that represents a physical quantity of interest.
The goal of adjoint-based error estimation is to
estimate, bound, or otherwise gain some information
about the functional error $\J(u) - \J(u_H)$.

\section{A discrete approach}

Let $R_h: \mathbb{R}^n \to \mathbb{R}^n$ denote the
system of $n$ potentially nonlinear algebraic
equations that arise from the Galerkin discretization
\eqref{eq:galerkin} on the fine space $V_h$ and
let $R_H: \mathbb{R}^N \to \mathbb{R}^N$ denote the
system of $N < n$ potentially nonlinear algebraic
equations that arise from the Galerkin discretization
\eqref{eq:galerkin} on the coarse space $V_H$.
Let $J_h: \mathbb{R}^n \to \mathbb{R}$ denote the
fine discretization of the functional on $V_h$ and
$J_H: \mathbb{R}^N \to \mathbb{R}$ denote the coarse
discretization of the functional on $V_H$.

Let $u_h \in \mathbb{R}^n$ denote the unique solution
to the fine discretization system $R_h(u_h) = 0$ and
$u_H$ denote the unique solution to the coarse
discretization system $R_H(u_H) = 0$. Let
$I^H_h: V_H \to V_h$ denote a prolongation from the
coarse discretization to the fine discretization
and denote the prolongation of the coarse solution
$u_H$ to the fine discretization $V_h$ as
$u^H_h = I^H_h u_H$.

The functional evaluated on the fine grid can be expanded
in a Talyor series centered at the prolongated
coarse-grid solution as
%
\begin{equation}
J_h(u_h) = J_h(u_h^H) +
\left[ \frac{\partial J_h}{\partial u_h} \biggr|_{u_h^H} \right]
(u_h - u_h^H).
\label{eq:functional_taylor}
\end{equation}
%
A similar Taylor expansion for the residual holds
%
\begin{equation}
R_h(u_h) = R_h(u_h^H) +
\left[ \frac{\partial R_h}{\partial u_h} \biggr|_{u_h^H} \right]
(u_h - u_h^H).
\label{eq:residual_taylor}
\end{equation}
%
Noting that $R_h(u_h) = 0$, we obtain the first
order approximation for the solution error between
the two discretizations:
\begin{equation}
(u_h - u_h^H) \approx
- \left[ \frac{\partial R_h}{\partial u_h} \biggr|_{u_h^H} \right]^{-1}
R_h(u_h^H).
\end{equation}
%
Thus, we can write the functional error between
the two discretizations as:
%
\begin{equation}
J_h(u_h) - J_h(u_h^H) \approx
-\left[ \frac{\partial J_h}{\partial u_h} \biggr|_{u_h^H} \right]
\left[ \frac{\partial R_h}{\partial u_h} \biggr|_{u_h^H} \right]^{-1}
R_h(u_h^H)
\end{equation}

We introduce the adjoint variable $z_h \in \mathbb{R}^n$ such that
%
\begin{equation}
z_h^T = 
-\left[ \frac{\partial J_h}{\partial u_h} \biggr|_{u_h^H} \right]
\left[ \frac{\partial R_h}{\partial u_h} \biggr|_{u_h^H} \right]^{-1},
\end{equation}
%
where $z_h$ is the solution to the linear system
%
\begin{equation}
\left[ \frac{\partial R_h}{\partial u_h} \biggr|_{u_h^H} \right]^T
z_h = 
\left[ \frac{\partial J_h}{\partial u_h} \biggr|_{u_h^H} \right]^T
\end{equation}
%
Physically, $z_h$ represents the sensitivity of the functional
$J_h$ to infinitesimal perturbations in the residual $R_h$.
Thus we can rewrite the functional error in terms of the
adjoint $z_h$ and the residual vector $R_h$ as
%
\begin{equation}
J_h(u_h) - J_h(u_h^H) \approx -z_h^T R_h(u_h^H).
\end{equation}

In practice, it may be undesirable to solve for the
dual solution on the fine discretization. It is common to
solve the dual solution on the coarse discretization and
prolong it to the fine discretization such that
$z_h^H = I_h^H z_H$ and write a functional error estimate as:
%
\begin{equation}
J_h(u_h) - J_h(u_h^H) \approx - \left[z_h^H \right]^T R_h(u_h^H).
\end{equation}

\section{A continuous approach}

\subsection{Taylor expansions with exact integral remainders}

Let $f: \mathbb{R} \to \mathbb{R}$ be a twice differentiable
function with first and second derivatives $f'$ and $f''$.
Let $u,w \in \mathbb{R}$. From the fundamental lemma of calculus
%
\begin{equation}
f(u+w) = f(u) + \int_u^{u+w} f'(t) \, \text{d} t.
\label{eq:function_taylor}
\end{equation}
%
The first derivative $f'(t)$ can be expressed as
%
\begin{equation}
f'(t) = \frac{d}{dt}
\left[(t-u-w) f'(t) \right] -
(t-u-w) f''(t)
\end{equation}
%
via the product rule. Equation \eqref{eq:function_taylor}
can then be rewritten as
%
\begin{equation}
f(u+w) = f(u) +
(t-u-w) f'(t) \biggr|_{t=u}^{t=u+w} -
\int_{u}^{u+w} (t-u-w) f''(t) \, \text{d} t,
\end{equation}
again using the fundamental lemma of calculus,
which reduces to
%
\begin{equation}
f(u+w) = f(u) +
w f'(u) + 
\int_{u}^{u+w} (u+w-t) f''(t) \, \text{d} t,
\end{equation}
%
upon evaluation. Introducing the change of variable
$t = u+sw$, where $\text{d}t = w \, \text{d}s$,
the equation can be rewritten as
%
\begin{equation}
f(u+w) = f(u) +
w f'(u)  +
\int_0^1 w^2 (1-s) f''(u + sw) \, \text{d} s.
\label{eq:function_remainder}
\end{equation}

\subsection{Extensions to functionals}

The arguments extend above extend to functionals
for sufficiently `well-behaved' functionals.
Let $u,v,w \in V$.
Let $\R(\cdot; \cdot)$ and $\J(\cdot)$ defined
previously be twice Fr\'{e}chet differentiable,
with first directional derivatives
%
\begin{equation}
\J'(u)\{w\} := \lim_{\epsilon \to 0}
\epsilon^{-1} \left[ \J(u + \epsilon w) - \J(u) \right],
\end{equation}
%
\begin{equation}
\R'(u;v)\{w\} := \lim_{\epsilon \to 0}
\epsilon^{-1} \left[ \R(u + \epsilon w; v) - \R(u;v) \right],
\end{equation}
and second directional derivatives
%
\begin{equation}
\J''(u)\{w,w\} := \lim_{\epsilon \to 0}
\epsilon^{-1} \left[ \J'(u + \epsilon w)\{w\} - \J'(u)\{w\} \right],
\end{equation}
%
\begin{equation}
\R''(u;v)\{w,w\} := \lim_{\epsilon \to 0}
\epsilon^{-1} \left[ \R'(u + \epsilon w; v)\{w\} - \R'(u;v)\{w\} \right].
\end{equation}
Then the following Taylor expansion identities hold:
%
\begin{equation}
\J(u+w) = \J(u) + \J'(u)\{w\} + \delta_{\J}(u)\{w\}
\label{eq:functional_taylor_int}
\end{equation}
%
\begin{equation}
\R(u+w;v) = \R(u;v) + \R'(u;v)\{w\} + \delta_{\R}(u;v)\{w\}
\label{eq:residual_taylor_int}
\end{equation}
%
where the remainders $\delta_{\J}$ and
$\delta_{\R}$ are expressed in integral form as
%
\begin{equation}
\delta_{\J}(u)\{w\} = \int_0^1 \J''(u+sw)\{w,w\}(1-s) \, \text{d} s,
\end{equation}
and
%
\begin{equation}
\delta_{\R}(u;v)\{w\} = \int_0^1 \R''(u+sw;v)\{w,w\}(1-s) \, \text{d} s,
\end{equation}
respectively.

\subsection{Error representation}

Let $u$ denote the exact solution to the
variational statement \eqref{eq:model}
and let $u_h$ denote the unique solution to the Galerkin
formulation \eqref{eq:galerkin}. Let the solution error
be denoted $e := u - u_h$. Let $z \in V$ be the unique
solution to the dual problem:
%
\begin{equation}
\R'(u_h; z)\{v\} = \J'(u_h)\{v\} \quad \forall \, v \in V.
\label{eq:dual}
\end{equation}
%
which is assumed to be well-posed.

The functional error can then be expressed as:
\begin{equation*}
\begin{aligned}
\J(u) - \J(u_h) &= \J(u_h+e) - \J(u_h) \\
&= \J'(u_h)\{e\} + \delta_{\J}(u_h)\{e\} \\
&= \R'(u_h; z)\{e\} + \delta_{\J}(u_h)\{e\} \\
&= \R(u; z) - \R(u_h; z) - \delta_{\R}(u_h;z)\{e\} + \delta_{\J}(u_h)\{e\} \\
&= 
\underbrace{-\R(u_h; z)}_{\text{discretization error}}
\underbrace{-\delta_{\R}(u_h;z)\{e\} + \delta_{\J}(u_h)\{e\}}
_{\text{linearization error}}
\end{aligned}
\end{equation*}
%
The first equality is due to the definition of
the solution error $e$, the second equality is due
to the expansion \eqref{eq:functional_taylor_int},
the third equality is due to the dual problem
\eqref{eq:dual}, the fourth equality is due to the
expansion \eqref{eq:residual_taylor_int}, and the
fifth equality holds since $\R(u;z) = 0 \quad \forall
z \in V$.

Note that the dual solution cannot be approximated
in the space $V_h$, otherwise $\R(u_h; v_h) = 0$ due
to Galerkin orthogonality. Additionally, note that the
error estimate $\J(u) - \J(u_h) \approx - \R(u_h; z - z_h)$
holds for all $z_h \in V_h$. This is useful during
error localization, where it is desirable to refine
in areas where the dual solution is not well approximated
rather than in areas where the dual solution is large.

\end{document}
